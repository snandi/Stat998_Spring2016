\section*{Data exploration} \label{Sec_Data}
\subsection*{Missing value imputation}
Before analyzing the data, the missing value for the variable ``Days\_In\_Milk'' was imputed by fitting a linear model between cow characteristics and ``Days\_In\_Milk''. The imputed value was $29.2126$. The details of this model fit is discussed in the Appendix. 

\subsection*{Analyzing cow characteristics}
Only one out of the 59 cows in the data-set have a ``parity'' of 4. This data point appears different from the rest of the sample. For example, in the plots between ``C15:0'' and ``parity'' a quadratic trend is evident, while between ``C18:1'' and ``parity'', a linearly decreasing trend is evident. 


Some of the cow characteristics are correlated to each other. For example, ``dry\_matter\_intake'' and ``parity'' have a correlation coefficient ($\rho$) of 0.631 (p-value $\sim \ 0$), for ``milk\_yield'' and ''dry\_matter\_intake'' $\rho = 0.529, (\text{ p-value } \sim \ 0$). The rest of the correlation coefficients are presented in table \ref{tab:corr_cow} in the Appendix. A similar pairwise correlation analysis was done between all the fatty acid concentrations, and none of them were statistically significant. The results are in table \ref{tab:corr_acid} in the Appendix. Hence, ``C15:0'' and ``C18:1'' could be analyzed independently, without having to consider confounding problems. 

\subsection*{Exploring the MIR data}
Although the MIR data provide the absorbance at 1060 wavelengths, around 10\% of them exhibit presence of peaks in the signals. The plot below (fig \ref{fig:mir}) displays the variability in the peaks of the MIR absorbance.
Clearly, absorption in wavelengths between 422 and 431, and between 815 and 882 show some variability in signal peaks. The rest of the wavelengths do not exhibit much features. 

