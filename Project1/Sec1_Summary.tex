\section*{Summary}\label{Sec_Summary}
In recent years mid-infrared spectroscopy (MIR) has been developed as a way to use spectral analysis to estimate various milk properties including the percentage of fat and of protein in milk. Researchers from UW Dairy Science Department are interested in assessing the performance of the MIR technique for estimating the concentrations of several particular fatty acids. They have obtained data on 59 cows from the UW herd for this study. Several cow characteristics such as parity, number of days the cows have been producing milk, weights, physical condition, etc., have been recorded. In addition, the absorption data at 1060 different wavelengths have been recorded, from the MIR results. In addition, the gas chromatography (GCR) results have been recorded for fat percentage. The goal is to build a model that uses the different wavelength absorption peaks from the MIR data, the cow and milk characteristics to predict the fat percentage of fatty acid concentrations C18:1 and C15:0. 

A hierarchical approach was adopted for building this predictive model. First, a multiple linear regression model was fit between the fatty acid concentrations and different cow and milk characteristics. Second, the residuals from this regression model was fit individually with the 1060 wavelength absorption data. The wavelenghts with p-values lower than 0.1 or the wavelengths in the top 10\% most variable ones were chosen for the third round. Third, a partial least square (PLS) regression model was fit between the residuals from step 1, and the factors estimated by PLS. The third model was fit with ``prediction'' as the goal, on 75\% of randomly chosen data. The PLS fit was used to predict the 25\% of the remaining data and the predictions were compared. 

There was several challenges when analyzing this data set. Very few wavelengths had any statistically significant predictive power for the fatty acid concentrations. Secondly, there were too few observations compared to the number of predictors. Hence, a hierarchical approach was adopted to build the predictive model. It is recommended that many more data points are collected and a similar analysis is repeated before the MIR data can be used to predict the fatty acid concentration levels. 

