\section*{Discussion and Recommendation} \label{Sec_Discussion}

There was several challenges when analyzing this data set. Very few wavelengths had any statistically significant predictive power for the fatty acid concentrations. The univariate p-values were mostly $> 0.1$. Secondly, there were too few observations compared to the number of predictors. Hence, a hierarchical approach was adopted to build the predictive model. It is recommended that many more data points are collected and a similar analysis is repeated before the MIR data can be used to predict the fatty acid concentration levels. 

\subsection*{Recommendation with Data set 2}
There are five cow and milk related variables: ``Days in milk'', ``Milk yield'', ``Fat pct'', ``Protein pct'', ``Lactose pct''. The goal is to suggest a procedure for selecting a sample of cows from this herd, so that it produces a robust estimate. 

\begin{enumerate}
\item Consider each cow as a p-variate data point $(p = 5)$. Next, estimate the multivariate median (conceptually similar to univariate median). A multivariate median ($L_1-\text{Median}$) can be estimated by the algorithm proposed by Vardi and Zhang in \cite{Vardi_Zhang_2000_PNAS}, where $L_1-\text{Median }$ $y_m$ is the minimizer of \[ \sum\limits_{i = 1}^n \|y_i - y_m \| \]
where $y_i \in \Real^p, \ i = 1,\dots,n$ and $\|u \| = \sqrt{\sum\limits_{j = 1}^p u_j^2}$. Here, the $L_1-\text{Median }$ is the vector $(167.71, 94.99, 3.45, 3.12, 4.81)$. 
\item Estimate the euclidean distance between each ``cow'' or (p-variate data point) and this $L_1-\text{Median }$. Below are five cows that are farthest from the median and five cows that are closest. 
% latex table generated in R 3.2.2 by xtable 1.8-0 package
% Tue Feb 23 01:46:38 2016
\begin{table}[H]
\centering
\begin{tabular}{rrrrrrrr}
  \hline
  SampleID & CowID & Days & Yield & Fat & Protein & Lactose & Dist \\ 
  \hline
  268 & 7061 & 661 &  85 & 3.38 & 3.30 & 4.85 & 493.39 \\ 
  190 & 6872 & 489 &  57 & 3.50 & 3.30 & 4.62 & 323.53 \\ 
  145 & 6735 & 425 &  73 & 2.90 & 3.09 & 4.57 & 258.23 \\ 
  223 & 6933 & 424 & 110 & 2.74 & 2.77 & 5.03 & 256.73 \\ 
  241 & 6988 & 422 &  62 & 3.67 & 3.55 & 4.78 & 256.42 \\ 
  196 & 6884 & 415 &  38 & 3.26 & 3.47 & 4.45 & 253.77 \\ 
  \hline
\end{tabular}
\caption{5 cows that are farthest from the $L_1-\text{Median }$}
\label{Tab:farthest}
\end{table}

% latex table generated in R 3.2.2 by xtable 1.8-0 package
% Tue Feb 23 01:48:13 2016
\begin{table}[H]
\centering
\begin{tabular}{rrrrrrrr}
  \hline
  SampleID & CowID & Days & Yield & Fat & Protein & Lactose & Dist \\ 
  \hline
  186 & 6862 & 172 &  99 & 2.65 & 3.35 & 4.84 & 5.93 \\ 
  199 & 6888 & 162 &  95 & 3.38 & 3.36 & 4.36 & 5.74 \\ 
  269 & 7062 & 165 &  95 & 3.69 & 3.38 & 4.94 & 2.74 \\ 
  175 & 6838 & 166 &  97 & 3.12 & 3.32 & 4.91 & 2.67 \\ 
  170 & 6822 & 166 &  95 & 3.85 & 2.99 & 4.66 & 1.77 \\ 
  573 & 7585 & 167 &  94 & 4.13 & 3.19 & 4.80 & 1.40 \\ 
  \hline
\end{tabular}
\caption{5 cows that are nearest from the $L_1-\text{Median }$}
\label{Tab:closest}
\end{table}

\item Based on this distance, choose the cows that are farthest from the median. 
\end{enumerate}
