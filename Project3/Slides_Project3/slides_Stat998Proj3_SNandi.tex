%\documentclass[10pt,dvipsnames,table, handout]{beamer} % To printout the slides without the animations
\documentclass[10pt,dvipsnames,table]{beamer} 
%\usetheme{Luebeck} 
%\usetheme{Madrid} 
%\usetheme{Marburg} 
%\usetheme{Warsaw} 
\usetheme{CambridgeUS}
%\setbeamercolor{structure}{fg=cyan!90!white}
%\setbeamercolor{normal text}{fg=white, bg=black}
\setbeamercolor{block title}{bg=red!80,fg=white}

%%%%%%%%%%%%%%%%%%%%%%%%%%%%%%%%%%%%%%%%%%%%%%%%%%%%%%%%%%%%%%%%%%%%%%
%% Input header file 
%%%%%%%%%%%%%%%%%%%%%%%%%%%%%%%%%%%%%%%%%%%%%%%%%%%%%%%%%%%%%%%%%%%%%%
../../../../TexScripts/HeaderfileTexSlides.tex
\usepackage[latin1]{inputenc}
\usepackage{times}
\usepackage{tikz}
\usetikzlibrary{arrows,shapes,shapes.geometric}

%\logo{\includegraphics[scale=0.4]{uwlogo_web_sm_fl_wht.png}}
%\logo{\includegraphics[width=\beamer@sidebarwidth,height=\beamer@headheight]{uwlogo_web_sm_fl_wht.png}}
%%%%%%%%%%%%%%%%%%%%%%%%%%%%%%%%%%%%%%%%%%%%%%%%%%%%%%%%%%%%%%%%%%%%%%
%% TITLE PAGE 
%%%%%%%%%%%%%%%%%%%%%%%%%%%%%%%%%%%%%%%%%%%%%%%%%%%%%%%%%%%%%%%%%%%%%%

\DeclarePairedDelimiter\ceil{\lceil}{\rceil}
\title[Callibrating spectroscopes]{Estimation of inter-instrument callibration function}
\author{Subhrangshu Nandi}
\institute[Stat 998]{Stat 998, Spring 2016 \\
Department of Statistics \\
University of Wisconsin-Madison}
\date{April 12, 2016}

\begin{document}
\setlength{\baselineskip}{16truept}
\setbeamertemplate{logo}{}

\frame{\maketitle}

%%%%%%%%%%%%%%%%%%%%%%%%%%%%%%%%%%%%%%%%%%%%%%%%%%%%%%%%%%%%%%%%%

%%%%%%%%%%%%%%%%%%%%%%%%%%%% Slide x %%%%%%%%%%%%%%%%%%%%%%%%%%%%
\begin{frame}
\frametitle{Client - The Townsend Lab}
Professor
\end{frame}
%%%%%%%%%%%%%%%%%%%%%%%%%%%%%%%%%%%%%%%%%%%%%%%%%%%%%%%%%%%%%%%%%

%%%%%%%%%%%%%%%%%%%%%%%%%%%% Slide x %%%%%%%%%%%%%%%%%%%%%%%%%%%%
\begin{frame}
\frametitle{Leaf Spectrometers}
\end{frame}
%%%%%%%%%%%%%%%%%%%%%%%%%%%%%%%%%%%%%%%%%%%%%%%%%%%%%%%%%%%%%%%%%

%%%%%%%%%%%%%%%%%%%%%%%%%%%% Slide x %%%%%%%%%%%%%%%%%%%%%%%%%%%%
\begin{frame}
\frametitle{Inter-instrument difference}
\vspace{-0.3cm}
\begin{center}
\begin{figure}
\includegraphics[scale=0.22, page=36]{../Plots/PairwisePlots_cal.pdf} \hspace{1cm}
\includegraphics[scale=0.22, page=37]{../Plots/PairwisePlots_cal.pdf} \\
\includegraphics[scale=0.22, page=206]{../Plots/PairwisePlots_cal.pdf} \hspace{1cm}
\includegraphics[scale=0.22, page=207]{../Plots/PairwisePlots_cal.pdf}
\end{figure}
\end{center}
\end{frame}
%%%%%%%%%%%%%%%%%%%%%%%%%%%%%%%%%%%%%%%%%%%%%%%%%%%%%%%%%%%%%%%%

%%%%%%%%%%%%%%%%%%%%%%%%%%%% Slide x %%%%%%%%%%%%%%%%%%%%%%%%%%%%
\begin{frame}
\frametitle{The spectroscopy workflow}
\tikzstyle{startstop} = [rectangle, rounded corners, minimum width=3cm, minimum height=1cm, text width=3cm, text centered, draw=black, fill=red!30]
\tikzstyle{io} = [trapezium, trapezium left angle=70, trapezium right angle=110, minimum width=3cm, minimum height=1cm, text width=3cm, text centered, draw=black, fill=blue!30]
\tikzstyle{process} = [rectangle, rounded corners, minimum width=3cm, minimum height=1cm, text width=3cm, text centered, draw=black, fill=orange!30]
\tikzstyle{decision} = [rectangle, rounded corners, minimum width=3cm, minimum height=1cm, text width=3cm, text centered, draw=black, fill=green!30]
\tikzstyle{arrow} = [thick,->,>=stealth]
\tikzstyle{line} = [thick, draw, -latex']
\tikzset{every cross out node/.append style={-, solid}}

\begin{figure}
\begin{tikzpicture}[node distance=2cm]
\node (traindata) [startstop] {Training data from ASD Probe};
\node (model) [io, below of=traindata] {Model fitting; PLS regression};
\node (pls) [process, below of=model] {PLS scores; Prediction coefficients};
\node (dec) [decision, below of=pls] {Species;\\ Nitrogen content; Cellulose; Carbon};

\draw [arrow] (traindata) -- (model);
\draw [arrow] (model) -- (pls);
\draw [arrow] (pls) -- (dec);

\pause
\node (data1) [startstop, left of=traindata, xshift=-2cm] {New data from ASD};
\path [line] (data1) |- (pls);

\pause
\node (data2) [startstop, right of=traindata, xshift=2cm] {New data from Other instruments};
\path [line] (data2) |- node[draw=red, cross out, line width=.5ex, minimum width=1.5ex, minimum height=1ex, anchor=center]{}(pls);
% node[very near start, yshift=0.5ex, xshift=-0.5ex,  font=\tiny] {4} (m0);

\pause
\node (cali) [io, right of=dec, xshift=2cm, yshift=0.5cm, fill=red!60] {Need \\ inter-instrument calibration!!};
\end{tikzpicture}
\end{figure}
\end{frame}
%%%%%%%%%%%%%%%%%%%%%%%%%%%%%%%%%%%%%%%%%%%%%%%%%%%%%%%%%%%%%%%%%

%%%%%%%%%%%%%%%%%%%%%%%%%%%% Slide x %%%%%%%%%%%%%%%%%%%%%%%%%%%%
\begin{frame}
\frametitle{Types of discrepancy}
\begin{columns}[t]
\begin{column}{0.6\textwidth}
\vspace{-0.3cm}
\begin{center}
\begin{figure}
\includegraphics[scale=0.4, page=10]{../Plots/PairwisePlots_withDiff_trunc.pdf} 
\end{figure}
\end{center}
\end{column}
\pause

\begin{column}{0.4\textwidth}
\vspace{1.5cm}
\begin{block}{Types of discrepancy}
\begin{enumerate}
\item Amplitude difference
\vspace{1cm}
\item Phase difference
\end{enumerate}
\end{block}
\end{column}
\end{columns}
\end{frame}
%%%%%%%%%%%%%%%%%%%%%%%%%%%%%%%%%%%%%%%%%%%%%%%%%%%%%%%%%%%%%%%%%

%%%%%%%%%%%%%%%%%%%%%%%%%%%% Slide x %%%%%%%%%%%%%%%%%%%%%%%%%%%%
\begin{frame}
\frametitle{Statistical problem}
\begin{block}{Eliminate phase variability}
Using curve registration
\end{block}
\vspace{1 cm}
\pause
\begin{block}{Estimate calibration function}
Using function on function regression (concurrent multiple regression)
\end{block}
\end{frame}
%%%%%%%%%%%%%%%%%%%%%%%%%%%%%%%%%%%%%%%%%%%%%%%%%%%%%%%%%%%%%%%%%

%%%%%%%%%%%%%%%%%%%%%%%%%%%% Slide x %%%%%%%%%%%%%%%%%%%%%%%%%%%%
\begin{frame}
\frametitle{Registration details}
\begin{enumerate}
\item Let $x(t)$ be the true curve; $y(t)$ be the curve to register. 
\item Let $h(t)$ be the warping function; $y[h(t)]$ be the registered curve.
\item $y[h(t)]$ and and $x(t)$ differ only in terms of amplitude variation, i.e., their values are proportional to one another across the range of $t$
\item Then, PCA of the following order two matrix should reveal essentially one component (smallest eigen value $\approx 0$) \\
\[C(h) = 
\begin{bmatrix}
\int \{x(t) \}^2dt & \int x(t) y[h(t)]dt \vspace{0.5cm} \\ 
\int x(t) y[h(t)]dt & \int \{y[h(t)]\}^2dt
\end{bmatrix}
\]
\item Choose $h(t)$ that minimizes the eigen value
\end{enumerate}
\end{frame}
%%%%%%%%%%%%%%%%%%%%%%%%%%%%%%%%%%%%%%%%%%%%%%%%%%%%%%%%%%%%%%%%%

%%%%%%%%%%%%%%%%%%%%%%%%%%%% Slide x %%%%%%%%%%%%%%%%%%%%%%%%%%%%
\begin{frame}
\frametitle{Estimation of the warping function}
\begin{enumerate}
\item $h(t)$ is a strictly {\emph{monotone increasing}} function; Hence {\emph{invertible}}.
\item Let $w(t) = \frac{h''(t)}{h'(t)}$ be the relative acceleration of the warping function. \\
This will be used to {\emph{constrain}} the warping
\item The objective function is 
\[ \text{MINEIG}_{\lambda}(h) = \text{MINEIG}(h) + \lambda \int \{w^{(m)}(t)\}^2 dt \]
which is similar to:
\[ F_{\lambda}(y,x|h) = \int \|y'(t) - x'\{h(t)\} \|^2 dt + \lambda \int \{w^{(m)}(t)\}^2 dt \]
Penalizing $w(t)$ ensures smoothness and monotonicity
\item Express $w(t)$ in terms of B-Spline expansion: 
\[w(t) = \sum \limits_{k=0}^{K} c_k B_k(t)\]
Solve for the coefficients $c_k$
\end{enumerate}
\end{frame}
%%%%%%%%%%%%%%%%%%%%%%%%%%%%%%%%%%%%%%%%%%%%%%%%%%%%%%%%%%%%%%%%%

%%%%%%%%%%%%%%%%%%%%%%%%%%%% Slide x %%%%%%%%%%%%%%%%%%%%%%%%%%%%
\begin{frame}
\frametitle{Functional regression framework}
\begin{enumerate}
\item Response: ASD $y$; Covariate: SE $w$; Abscissa: Wavelength $x$
\item Basis expansion representation of $w$:
\[w(x) = \sum \limits_{j=0}^{q} c_j z_j(x)\]
Choice of basis: B-spline or PCA basis
\item For leaf $i$, the concurrent multiple regression model is:
\[ y_i(x) = \sum \limits_{j=0}^{q} z_{ij}(x)\beta_j(x) + \epsilon_i(x) \]
\item In matrix form
\[ \mathbf{y}(x) = \mathbf{Z}(x)\mathbf{\beta}(x) + \mathbf{\epsilon}(x)\]
\end{enumerate}
\end{frame}
%%%%%%%%%%%%%%%%%%%%%%%%%%%%%%%%%%%%%%%%%%%%%%%%%%%%%%%%%%%%%%%%%

%%%%%%%%%%%%%%%%%%%%%%%%%%%% Slide x %%%%%%%%%%%%%%%%%%%%%%%%%%%%
\begin{frame}
\frametitle{Project workflow}
\begin{enumerate}
\item Eliminate phase variability
\item Choose the appropriate basis expansion for the functions
\item Fit ASD spectroscopic values to SE values using functional regression
\item Calibration function: Convolution of warping function from registration and fit from the regression model
\item Estimate and compare prediction error before and after calibration 
\item Use ``species'' as a random effect if desired
\end{enumerate}
\end{frame}
%%%%%%%%%%%%%%%%%%%%%%%%%%%%%%%%%%%%%%%%%%%%%%%%%%%%%%%%%%%%%%%%%

%%%%%%%%%%%%%%%%%%%%%%%%%%%% Slide x %%%%%%%%%%%%%%%%%%%%%%%%%%%%
\begin{frame}
\frametitle{Questions and suggestions please}

Thank you very much!

\end{frame}
%%%%%%%%%%%%%%%%%%%%%%%%%%%%%%%%%%%%%%%%%%%%%%%%%%%%%%%%%%%%%%%%%

%%%%%%%%%%%%%%%%%%%%%%%%%%%% Slide x %%%%%%%%%%%%%%%%%%%%%%%%%%%%
\begin{frame}
\frametitle{Reference}
{\footnotesize{
    \bibliographystyle{apalike}
    \bibliography{bibTex_Reference}
}}
\end{frame}
%%%%%%%%%%%%%%%%%%%%%%%%%%%%%%%%%%%%%%%%%%%%%%%%%%%%%%%%%%%%%%%%%
\end{document}

