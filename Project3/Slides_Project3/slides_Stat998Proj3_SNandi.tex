%\documentclass[10pt,dvipsnames,table, handout]{beamer} % To printout the slides without the animations
\documentclass[10pt,dvipsnames,table]{beamer} 
%\usetheme{Luebeck} 
%\usetheme{Madrid} 
%\usetheme{Marburg} 
%\usetheme{Warsaw} 
\usetheme{CambridgeUS}
%\setbeamercolor{structure}{fg=cyan!90!white}
%\setbeamercolor{normal text}{fg=white, bg=black}
\setbeamercolor{block title}{bg=red!80,fg=white}

%%%%%%%%%%%%%%%%%%%%%%%%%%%%%%%%%%%%%%%%%%%%%%%%%%%%%%%%%%%%%%%%%%%%%%
%% Input header file 
%%%%%%%%%%%%%%%%%%%%%%%%%%%%%%%%%%%%%%%%%%%%%%%%%%%%%%%%%%%%%%%%%%%%%%
../../../../TexScripts/HeaderfileTexSlides.tex
\usepackage[latin1]{inputenc}
\usepackage{times}
\usepackage{tikz}
\usetikzlibrary{arrows,shapes,shapes.geometric}

%\logo{\includegraphics[scale=0.4]{uwlogo_web_sm_fl_wht.png}}
%\logo{\includegraphics[width=\beamer@sidebarwidth,height=\beamer@headheight]{uwlogo_web_sm_fl_wht.png}}
%%%%%%%%%%%%%%%%%%%%%%%%%%%%%%%%%%%%%%%%%%%%%%%%%%%%%%%%%%%%%%%%%%%%%%
%% TITLE PAGE 
%%%%%%%%%%%%%%%%%%%%%%%%%%%%%%%%%%%%%%%%%%%%%%%%%%%%%%%%%%%%%%%%%%%%%%

\DeclarePairedDelimiter\ceil{\lceil}{\rceil}
\title[Callibrating spectroscopes]{Estimation of inter-instrument callibration function}
\author{Subhrangshu Nandi}
\institute[Stat 998]{Stat 998, Spring 2016 \\
Department of Statistics \\
University of Wisconsin-Madison}
\date{April 12, 2016}

\begin{document}
\setlength{\baselineskip}{16truept}
\setbeamertemplate{logo}{}

\frame{\maketitle}

\begin{frame}
\frametitle{Client - The Townsend Lab}
Professor
\end{frame}

\begin{frame}
\frametitle{Leaf Spectrometers}
\end{frame}

\begin{frame}
\frametitle{Inter-instrument difference}
\end{frame}

%% \begin{frame}
%% \frametitle{The spectroscopy workflow}
%% \tikzstyle{format} = [draw, thin, fill=blue!20]
%% \tikzstyle{medium} = [ellipse, draw, thin, fill=green!20, minimum height=2.5em]

%% \begin{figure}
%% \begin{tikzpicture}[node distance=3cm, auto]
%% %\begin{tikzpicture}[node distance=3cm, auto,>=latex', thick]
%%     % We need to set at bounding box first. Otherwise the diagram
%%     % will change position for each frame.
%%     \path[use as bounding box] (-1,0) rectangle (10,-2);
%%     \path[->]<1-> node[format] (tex) {.tex file};
%%     \path[->]<2-> node[format, right of=tex] (dvi) {.dvi file}
%%                   (tex) edge node {\TeX} (dvi);
%%     \path[->]<3-> node[format, right of=dvi] (ps) {.ps file}
%%                   node[medium, below of=dvi] (screen) {screen}
%%                   (dvi) edge node {dvips} (ps)
%%                         edge node[swap] {xdvi} (screen);
%%     \path[->]<4-> node[format, right of=ps] (pdf) {.pdf file}
%%                   node[medium, below of=ps] (print) {printer}
%%                   (ps) edge node {ps2pdf} (pdf)
%%                        edge node[swap] {gs} (screen)
%%                        edge (print);
%%     \path[->]<5-> (pdf) edge (screen)
%%                         edge (print);
%%     \path[->, draw]<6-> (tex) -- +(0,1) -| node[near start] {pdf\TeX} (pdf);
%% \end{tikzpicture}
%% \end{figure}
%% \end{frame}

\begin{frame}
\frametitle{The spectroscopy workflow}
\tikzstyle{startstop} = [rectangle, rounded corners, minimum width=3cm, minimum height=1cm, text width=3cm, text centered, draw=black, fill=red!30]
\tikzstyle{io} = [trapezium, trapezium left angle=70, trapezium right angle=110, minimum width=3cm, minimum height=1cm, text width=3cm, text centered, draw=black, fill=blue!30]
\tikzstyle{process} = [rectangle, rounded corners, minimum width=3cm, minimum height=1cm, text width=3cm, text centered, draw=black, fill=orange!30]
\tikzstyle{decision} = [rectangle, rounded corners, minimum width=3cm, minimum height=1cm, text width=3cm, text centered, draw=black, fill=green!30]
\tikzstyle{arrow} = [thick,->,>=stealth]
\tikzstyle{line} = [draw, -latex']

\begin{figure}
\begin{tikzpicture}[node distance=2cm]
\node (traindata) [startstop] {Training data from ASD Probe};
\node (model) [io, below of=traindata] {Model fitting; PLS regression};
\node (pls) [process, below of=model] {PLS scores; Prediction coefficients};
\node (dec) [decision, below of=pls] {Species;\\ Nitrogen content; Cellulose; Carbon};

\draw [arrow] (traindata) -- (model);
\draw [arrow] (model) -- (pls);
\draw [arrow] (pls) -- (dec);

\pause
\node (data1) [startstop, left of=traindata, xshift=-2cm] {New data from ASD};
\path [line] (data1) |- (pls);

\pause
\node (data2) [startstop, right of=traindata, xshift=2cm] {New data from Other instruments};
\path [line] (data2) |- node[above]{\LARGE{no}}  (pls);

\end{tikzpicture}
\end{figure}
\end{frame}

\begin{frame}
\frametitle{Questions and suggestions please}

Thank you very much!

\end{frame}

%%%%%%%%%%%%%% Slide x %%%%%%%%%%%%%%
\begin{frame}
\frametitle{Reference}
{\footnotesize{
    \bibliographystyle{apalike}
    \bibliography{bibTex_Reference}
}}
\end{frame}

\end{document}

